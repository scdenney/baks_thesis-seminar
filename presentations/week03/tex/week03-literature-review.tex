% Week 3: Conducting a Literature Review
% BA-Eindwerkstuk Seminar - Leiden University
% Dr. Steven Denney

\documentclass[aspectratio=169,11pt]{beamer}

% ============================================
% Packages
% ============================================
\usepackage{tikz}
\usepackage{graphicx}
\usepackage{hyperref}
\usepackage{fontspec}
\usepackage{booktabs}

% Load custom theme
\usepackage{beamerthemethesis}

% ============================================
% Metadata
% ============================================
\title{BA-Eindwerkstuk Seminar}
\subtitle{Week 3: Conducting a Literature Review}
\author{Dr. Steven Denney}
\institute{Korean Studies \\ Leiden University}
\date{February 20, 2026}

% ============================================
% Document
% ============================================
\begin{document}

% --------------------------------------------
% Title Slide
% --------------------------------------------
\begin{frame}[plain,noframenumbering]
    \titlepage
\end{frame}

% ============================================
\section{Review from Week 2}
% ============================================

\begin{frame}{Roundtable: Your Question, Motivation, and Data}
    \textbf{Each of you, briefly share:}
    \vspace{0.3cm}
    \begin{enumerate}
        \item \textbf{Your research question} (one sentence)
        \item \textbf{Your motivation} --- why does this matter? What is the problem or gap? (2--3 sentences)
        \item \textbf{Your data} --- what type of data do you have, plan to collect, or think you might need?
    \end{enumerate}
    \vspace{0.5cm}
    \begin{block}{Why revisit this?}
        Your question, motivation, and data are the foundation of everything that follows. The literature review connects all three.
    \end{block}
\end{frame}

\begin{frame}{Recall: The Question, Problem, and Analysis}
    Three things you should be able to state clearly:
    \vspace{0.3cm}
    \begin{enumerate}
        \item \textbf{What is your question?}
        \item \textbf{What is your problem?} (the gap or puzzle in the existing literature)
        \item \textbf{What are your sources, and how will you analyze them?}
    \end{enumerate}
    \vspace{0.5cm}
    Today we focus on \#2: how the literature review helps you \textit{define} and \textit{justify} your research problem.
\end{frame}

% ============================================
\section{What Is a Literature Review?}
% ============================================

\begin{frame}{Today's Agenda}
    \begin{enumerate}
        \item Review of Week 2 and roundtable
        \item Purpose of a literature review
        \item Types of literature reviews
        \item Reflect: what kind of review will you write?
        \item Best practices for searching, reading, and writing
        \item Example: a working paper on national identity in Taiwan
        \item Exercise: annotated bibliography and draft outline
    \end{enumerate}
\end{frame}

\begin{frame}{Purpose of a Literature Review}
    A literature review is \textbf{not} a summary of everything you have read.
    \vspace{0.3cm}

    It is a \textit{critical analysis} of existing scholarship that:
    \vspace{0.3cm}
    \begin{itemize}
        \item \textbf{Situates your research in context:} Shows how your study extends, refines, or challenges existing work.
        \item \textbf{Identifies gaps in knowledge:} Reveals contradictions, unresolved questions, or overlooked areas in prior studies.
        \item \textbf{Motivates empirical strategies:} Shows what methods and empirical strategies are available and (sometimes) why these should be adjusted.
        \item \textbf{Establishes credibility and relevance:} Demonstrates you have built upon a recognized foundation of scholarship.
    \end{itemize}
\end{frame}

\begin{frame}{The Lit Review in Your Thesis}
    \begin{enumerate}
        \item \textbf{Introduction} --- question, motivation, overview
        \item \textbf{Literature Review} $\leftarrow$ \textit{you are here}
        \begin{itemize}
            \item Situate the question in existing scholarship
            \item Identify the gap your thesis addresses
            \item Build toward your analytical framework
        \end{itemize}
        \item \textbf{Methodology / Analytical Framework}
        \item \textbf{Findings}
        \item \textbf{Discussion and Conclusion}
    \end{enumerate}
    \vspace{0.3cm}
    \begin{alertblock}{Key Point}
        The lit review is the bridge between your question and your method. It tells the reader \textit{why} your question matters and \textit{how} others have approached similar problems.
    \end{alertblock}
\end{frame}

% ============================================
\section{Types of Literature Reviews}
% ============================================

\begin{frame}{Argumentative Review}
    \begin{itemize}
        \item \textbf{Purpose:} Critically examine a body of literature to support or refute an existing argument or assumption.
        \item \textbf{Approach:} Selectively include sources to develop a stance or ``contrarian'' viewpoint.
        \item \textbf{Watch out for:} Potential bias---argumentative reviews can overemphasize sources that favor one side.
    \end{itemize}
\end{frame}

\begin{frame}{Integrative Review}
    \begin{itemize}
        \item \textbf{Purpose:} Summarize and synthesize past research to generate new perspectives.
        \item \textbf{Approach:} Include all relevant studies addressing the same research question or hypothesis.
        \item \textbf{Key value:} Creates or refines frameworks and theories.
    \end{itemize}
    \vspace{0.5cm}
    \textit{Most common type in the social sciences.}
\end{frame}

\begin{frame}{Historical Review}
    \begin{itemize}
        \item \textbf{Purpose:} Trace the chronological development of a theory, concept, or issue.
        \item \textbf{Approach:} Identify how a topic or scholarly conversation evolved through time.
        \item \textbf{Key value:} Provides historical context and identifies trends or turning points to inform future research.
    \end{itemize}
\end{frame}

\begin{frame}{Methodological Review}
    \begin{itemize}
        \item \textbf{Purpose:} Focus on research methods rather than findings.
        \item \textbf{Approach:} Compare how studies collect and analyze data, exposing strengths, weaknesses, or patterns.
        \item \textbf{Key value:} Highlights new approaches, ethical considerations, and overlooked techniques in the field.
    \end{itemize}
\end{frame}

\begin{frame}{Systematic Review}
    \begin{itemize}
        \item \textbf{Purpose:} Provide a comprehensive and unbiased summary of all available evidence on a focused research question.
        \item \textbf{Approach:} Follows predetermined, transparent criteria to search, evaluate, and synthesize existing research.
        \item \textbf{Typical in:} Clinical and health-related fields, but increasingly adopted in the social sciences.
    \end{itemize}
\end{frame}

\begin{frame}{Theoretical Review}
    \begin{itemize}
        \item \textbf{Purpose:} Examine how theory or theories have evolved to explain a phenomenon.
        \item \textbf{Approach:} Explore existing conceptual frameworks; identify gaps or inconsistencies; propose new hypotheses.
        \item \textbf{Key value:} Helps establish the theoretical basis for your own study and points to directions for further research.
    \end{itemize}
\end{frame}

\begin{frame}{Types of Literature Reviews (Recap)}
    \footnotesize
    \begin{enumerate}
        \item \textbf{Argumentative Review:} Support or refute an existing argument; watch for bias.
        \item \textbf{Integrative Review:} Summarize and synthesize past research to propose new perspectives.
        \item \textbf{Historical Review:} Trace the evolution of theories or topics over time.
        \item \textbf{Methodological Review:} Focus on methods and approaches used, not just findings.
        \item \textbf{Systematic Review:} Formal, exhaustive search with strict criteria; common in clinical/social sciences.
        \item \textbf{Theoretical Review:} Examine how theories emerge and interrelate in your field.
    \end{enumerate}
    \vspace{0.3cm}
    \textit{You will mix and match depending on your question, sources, and methods.}
\end{frame}

% ============================================
\section{Reflect}
% ============================================

\begin{frame}{What About You?}
    \vspace{1cm}
    \begin{itemize}
        \item What is your ``literature''?
        \item What kind of sources will you read?
        \item What kind of review(s) will you do?
    \end{itemize}
\end{frame}

\begin{frame}[plain,noframenumbering]
    \vspace{2.5cm}
    \begin{center}
        {\color{AccentBlue}\Huge\bfseries Break}
        \vspace{0.3cm}

        {\color{DustyRose}\rule{3cm}{0.8pt}}
        \vspace{0.5cm}

        {\color{TextMuted}\large 10 minutes}
    \end{center}
\end{frame}

% ============================================
\section{Best Practices}
% ============================================

\begin{frame}{Preparation}
    \begin{itemize}
        \item \textbf{Clarify your research question:} Understand your main goal and how it fits existing studies.
        \item \textbf{Conduct targeted searches:} Use library databases, journal indexes, citation tracking, and curated bibliographies. Make good use of bibliographies of representative studies!
        \item \textbf{Collect and organize sources:} Gather essential studies (the ``must-reads'') and categorize them thematically or chronologically. Work back from these.
    \end{itemize}
\end{frame}

\begin{frame}{Organization}
    \begin{itemize}
        \item \textbf{Chronological Structure:} Good if developments clearly follow a timeline, but not always necessary.
        \item \textbf{Thematic (Conceptual) Structure:} Group studies by themes, subtopics, or shared concepts.
        \item \textbf{Methodological Structure:} Compare/contrast research designs, samples, and data analysis techniques.
        \item \textbf{Combination Approach:} Often, reviews combine two or more structures (e.g., chronological then thematic).
    \end{itemize}
\end{frame}

\begin{frame}{Reading}
    \begin{itemize}
        \item \textbf{Summarize and evaluate:} Go beyond summarizing results; critique scope, evidence, and limitations.
        \item \textbf{Keep an annotated bibliography:} Include short, critical summaries and keywords for each source.
        \item \textbf{Track conceptual links:} Visual connections using digital or physical tools (tables, mind maps).
        \item \textbf{Manage references:} Use citation software (\textbf{Zotero}, Mendeley, EndNote) for organization.
    \end{itemize}
\end{frame}

\begin{frame}{Writing Tips}
    \begin{itemize}
        \item \textbf{Integrate primary and secondary sources:} Bring in direct evidence where relevant, but maintain focus on \textit{analyzing prior work}.
        \item \textbf{Look for patterns or consensus:} Determine convergences among studies, noting the main points of agreement.
        \item \textbf{Note conflicting perspectives:} Highlight where scholars disagree; show why it matters to your project.
        \item \textbf{Revisit and refine:} Reevaluate your sources and add new references as your own study evolves.
    \end{itemize}
\end{frame}

% ============================================
\section{Example}
% ============================================

\begin{frame}{Example: National Identity in Taiwan}
    \textbf{Paper:} ``Measuring National Identity with Conjoint Experiments: The Case of Taiwan'' (Denney, Steinhardt, and Qi, working paper)
    \vspace{0.5cm}

    \textbf{Question:} How should scholars measure national identity? Can conjoint experiments reveal trade-offs that conventional surveys miss?
    \vspace{0.3cm}

    \textbf{Problem:} Standard survey batteries (e.g., the ISSP) ask people to rate identity criteria one at a time. This produces ceiling effects and tells us nothing about how people \textit{weigh} competing criteria against each other.
\end{frame}

\begin{frame}{How the Lit Review Is Structured}
    The review is organized \textbf{thematically}, with each theme building toward the research gap:
    \vspace{0.3cm}
    \begin{enumerate}
        \item \textbf{National identity as multidimensional} --- Establishes that scholars have long distinguished ascriptive, civic, and voluntarist dimensions of identity (Kohn, Smith, Shulman, Bonikowski).
        \item \textbf{Democracy and national identity} --- Reviews theory arguing democracy and nationalism are intertwined (Nodia, Greenfeld, Linz); connects this to the Taiwan case.
        \item \textbf{Limitations of existing measurement} --- Critiques the ISSP battery: ceiling effects, no trade-offs, treats dimensions as independent. \textit{This is the gap.}
        \item \textbf{The conjoint solution} --- Proposes an alternative method and explains why it addresses the gap.
    \end{enumerate}
\end{frame}

\begin{frame}{What Makes This Lit Review Work?}
    \begin{itemize}
        \item \textbf{It is thematic, not source-by-source.} The review is organized around ideas, not a list of ``Scholar A says X, Scholar B says Y.''
        \item \textbf{Each theme serves a purpose.} Theme 1 establishes the concept. Theme 2 develops the theory. Theme 3 identifies the gap. Theme 4 motivates the method.
        \item \textbf{It builds an argument.} By the end of the review, the reader understands \textit{why} the study is needed.
        \item \textbf{It mixes review types.} Integrative (synthesizing identity scholarship) + methodological (critiquing how identity has been measured) + theoretical (connecting democracy and nationalism).
    \end{itemize}
\end{frame}

\begin{frame}{A Simple Template}
    You can use a similar structure for your thesis:
    \vspace{0.3cm}
    \begin{enumerate}
        \item \textbf{What is known?} --- Summarize the key scholarship on your topic.
        \item \textbf{What is debated or unresolved?} --- Identify disagreements, gaps, or limitations in prior work.
        \item \textbf{What is missing?} --- State the specific gap your research addresses.
        \item \textbf{How will you fill it?} --- Preview your approach (this leads into your methodology chapter).
    \end{enumerate}
    \vspace{0.3cm}
    \begin{exampleblock}{Remember}
        The literature review is not a separate chapter that exists on its own. It connects your question to your method. Every source you include should earn its place.
    \end{exampleblock}
\end{frame}

% ============================================
\section{Exercise \& Next Steps}
% ============================================

\begin{frame}{Exercise}
    \textbf{Generate an annotated bibliography* and a draft outline of the literature review.}

    {\color{DustyRose}\textbf{Due to Brightspace by 23:59 on February 26 (Thursday).}}
    \vspace{0.3cm}

    You should use this in your revised proposal.
    \vspace{0.3cm}

    \textbf{*Min. of 5 sources with the following annotations:}
    \begin{enumerate}
        \item \textit{Full Citation}: Provide the complete reference (Chicago Style) so readers can locate the source easily.
        \item \textit{Summary of Main Ideas}: Briefly outline the source's central argument(s), key findings, or themes ($\sim$2 sentences).
        \item \textit{Relevance or Evaluation}: Note the source's strengths, weaknesses, or importance for your research topic, indicating how it is connected to the research question ($\sim$2 sentences).
    \end{enumerate}
\end{frame}

\begin{frame}{Looking Ahead}
    \textbf{Week 4: Analysis and Reporting -- Empirical Strategies for Data Interpretation}
    \begin{itemize}
        \item Overview of analysis methods: qualitative and quantitative
        \item Structuring the findings section
        \item Connecting findings to the research question and literature review
    \end{itemize}
    \vspace{0.5cm}
    \textbf{Key Dates:}
    \begin{itemize}
        \item Assignment \#1 due: March 13, 2026
        \item Start thinking about your supervisor and schedule a meeting
    \end{itemize}
    \vspace{0.3cm}
    \begin{block}{Course Website}
        \url{https://scdenney.github.io/baks_thesis-seminar/}
    \end{block}
\end{frame}

\begin{frame}{Resources}
    USC Library's overview of literature review types:\\
    \url{https://libguides.usc.edu/writingguide/literaturereview}
    \vspace{0.5cm}

    Grant, Maria J., and Andrew Booth. ``A Typology of Reviews: An Analysis of 14 Review Types and Associated Methodologies.'' \textit{Health Information \& Libraries Journal} 26, no. 2 (2009): 91--108. \url{https://doi.org/10.1111/j.1471-1842.2009.00848.x}
    \vspace{0.5cm}

    Leiden University `Subject guide'* on a systematic literature review:\\
    \url{https://www.library.universiteitleiden.nl/subject-guides/systematic-reviews}
    \vspace{0.3cm}

    {\small * Many different resources are found in the university's subject guides!}
\end{frame}

\end{document}
