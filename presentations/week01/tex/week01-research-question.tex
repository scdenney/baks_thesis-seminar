% Week 1: Developing a Research Question and Identifying the Research Problem
% BA-Eindwerkstuk Seminar - Leiden University
% Dr. Steven Denney

\documentclass[aspectratio=169,11pt]{beamer}

% ============================================
% Packages
% ============================================
\usepackage{tikz}
\usepackage{graphicx}
\usepackage{hyperref}
\usepackage{fontspec}
\usepackage{booktabs}

% Load custom theme
\usepackage{beamerthemethesis}

% ============================================
% Metadata
% ============================================
\title{BA-Eindwerkstuk Seminar}
\subtitle{Week 1: Developing a Research Question and Identifying the Research Problem}
\author{Dr. Steven Denney}
\institute{Korean Studies \\ Leiden University}
\date{February 6, 2026}

% ============================================
% Document
% ============================================
\begin{document}

% --------------------------------------------
% Title Slide
% --------------------------------------------
\begin{frame}[plain,noframenumbering]
    \titlepage
\end{frame}

% --------------------------------------------
% Agenda
% --------------------------------------------
\begin{frame}{Today's Agenda}
    \begin{enumerate}
        \item Introductions
        \item Boot camp overview
        \item What is a research question?
        \item Finding your problem in the literature
        \item Thesis structure
        \item Workshop: Research question development
    \end{enumerate}
    \vspace{0.5cm}
    \begin{block}{Required Reading}
        Introduction and Part I of Mullaney \& Rea, \textit{Where Research Begins} (2022)
    \end{block}
\end{frame}

% ============================================
\section{Introductions}
% ============================================

\begin{frame}{Who's Who}
    \textbf{About me:}
    \begin{itemize}
        \item Assistant Professor, Korean Studies \& International Relations
        \item Research: Korean studies, computational social science, comparative politics
        \item \texttt{s.c.denney@hum.leidenuniv.nl}
    \end{itemize}
    \vspace{0.5cm}
    \textbf{About you:}
    \begin{itemize}
        \item Your name
        \item Your research interest (topic area)
        \item What do you hope to get from this seminar?
    \end{itemize}
\end{frame}

% ============================================
\section{Boot Camp Overview}
% ============================================

\begin{frame}{The First Four Weeks}
    The ``boot camp'' provides structured training before independent work with supervisors.
    \vspace{0.3cm}
    \begin{table}
        \small
        \begin{tabular}{lll}
            \toprule
            \textbf{Week} & \textbf{Date} & \textbf{Topic} \\
            \midrule
            1 & Feb. 06 & Developing a Research Question \\
            2 & Feb. 13 & Data and Sources (FAIR Principles) \\
            3 & Feb. 20 & Conducting a Literature Review \\
            4 & Feb. 27 & Analysis and Reporting \\
            \bottomrule
        \end{tabular}
    \end{table}
    \vspace{0.3cm}
    \begin{alertblock}{Key Deadline}
        Assignment \#1 (Revised Proposal) due \textbf{March 13, 2026}
    \end{alertblock}
\end{frame}

\begin{frame}{Today's Objectives}
    By the end of this session, you will be able to:
    \vspace{0.3cm}
    \begin{enumerate}
        \item \textbf{Define} what makes a strong research question
        \item \textbf{Distinguish} between a topic, a question, and a research problem
        \item \textbf{Identify} research gaps in existing literature
        \item \textbf{Outline} the basic structure of an academic thesis
        \item \textbf{Draft} a preliminary research question and problem statement
    \end{enumerate}
\end{frame}

% ============================================
\section{What is a Research Question?}
% ============================================

\begin{frame}{Topics Are Not Questions}
    \begin{columns}[T]
        \begin{column}{0.5\textwidth}
            \textbf{A topic is:}
            \begin{itemize}
                \item A subject area
                \item Broad and general
                \item A starting point
            \end{itemize}
            \vspace{0.3cm}
            \textit{Example:} ``Korean nationalism''
        \end{column}
        \begin{column}{0.5\textwidth}
            \textbf{A research question is:}
            \begin{itemize}
                \item Specific and focused
                \item Open-ended (not yes/no)
                \item Answerable through research
            \end{itemize}
            \vspace{0.3cm}
            \textit{Example:} ``How do South Korean history textbooks represent national identity in their coverage of the colonial period?''
        \end{column}
    \end{columns}
\end{frame}

\begin{frame}{Characteristics of Strong Research Questions}
    \begin{enumerate}
        \item \textbf{Specific} --- Clearly defined scope and boundaries
        \item \textbf{Open-ended} --- Cannot be answered with yes/no
        \item \textbf{Researchable} --- Can be investigated with available evidence
        \item \textbf{Relevant} --- Contributes to scholarly understanding
        \item \textbf{Personal} --- Connects to your genuine interests
    \end{enumerate}
    \vspace{0.5cm}
    \begin{exampleblock}{Mullaney \& Rea's Key Insight}
        ``Your question is your compass. It will guide every decision you make.'' (p. 20)
    \end{exampleblock}
\end{frame}

\begin{frame}{What to Avoid}
    \begin{alertblock}{Problematic Question Types}
        \begin{itemize}
            \item \textbf{Leading questions} --- Assume the answer (``Why did X fail?'')
            \item \textbf{Loaded questions} --- Contain value judgments (``Why is K-pop so influential?'')
            \item \textbf{Causal questions} --- Often too ambitious for BA thesis (``How did X cause Y?'')
            \item \textbf{Yes/no questions} --- Too narrow (``Did the IMF crisis affect Korea?'')
            \item \textbf{Jargon-heavy questions} --- Unclear to non-specialists
        \end{itemize}
    \end{alertblock}
    \vspace{0.3cm}
    \textbf{Tip:} Use exploratory language: ``How,'' ``In what ways,'' ``What role does...''
\end{frame}

\begin{frame}{Why Start with Yourself?}
    \begin{block}{Self-Evidence (Mullaney \& Rea)}
        The best research questions emerge from your own genuine curiosity and interests---not from trying to find a ``gap'' in the literature first.
    \end{block}
    \vspace{0.5cm}
    \textbf{Ask yourself:}
    \begin{itemize}
        \item What puzzles you about Korea?
        \item What have you encountered that you want to understand better?
        \item What topics do you find yourself returning to?
        \item What makes you curious, confused, or even frustrated?
    \end{itemize}
    \vspace{0.3cm}
    \textit{Your passion sustains you through the research process.}
\end{frame}

\begin{frame}{Examples: Good vs. Problematic Questions}
    \footnotesize
    \begin{table}
        \begin{tabular}{p{5.5cm}p{5.5cm}}
            \toprule
            \textbf{Problematic} & \textbf{Better} \\
            \midrule
            ``Why did North Korea develop nuclear weapons?'' (causal, too broad) & ``How do North Korean state media frame the country's nuclear program as a defensive necessity?'' \\
            \addlinespace
            ``Is K-pop popular?'' (yes/no, obvious) & ``In what ways do K-pop fan communities in Europe construct Korean cultural identity?'' \\
            \addlinespace
            ``Korean feminism'' (topic, not question) & ``How do South Korean feminist activists use social media to mobilize support for legislative reform?'' \\
            \addlinespace
            ``Did the comfort women movement succeed?'' (loaded, yes/no) & ``What strategies have comfort women advocacy groups used to gain international recognition since 2010?'' \\
            \bottomrule
        \end{tabular}
    \end{table}
\end{frame}

% ============================================
\section{Finding Your Problem}
% ============================================

\begin{frame}{From Question to Problem}
    \begin{block}{The Research Problem}
        A research problem identifies \textit{what is missing, unclear, or contested} in our current understanding of a topic. It justifies \textit{why} your research matters.
    \end{block}
    \vspace{0.5cm}
    \textbf{Types of research problems:}
    \begin{itemize}
        \item \textbf{Empirical} --- We lack evidence about X
        \item \textbf{Theoretical} --- Existing frameworks don't explain X
        \item \textbf{Methodological} --- Current approaches overlook X
        \item \textbf{Normative} --- There is disagreement about how to interpret X
    \end{itemize}
\end{frame}

\begin{frame}{Identifying Gaps in the Literature}
    \begin{columns}[T]
        \begin{column}{0.5\textwidth}
            \textbf{Common gap types:}
            \begin{itemize}
                \item Understudied populations
                \item Unexplored time periods
                \item Missing perspectives
                \item Methodological limitations
                \item Theoretical blind spots
            \end{itemize}
        \end{column}
        \begin{column}{0.5\textwidth}
            \textbf{How to find gaps:}
            \begin{itemize}
                \item Read recent literature reviews
                \item Look at ``future research'' sections
                \item Compare Korean and English scholarship
                \item Notice what's \textit{not} being discussed
                \item Talk to your supervisor
            \end{itemize}
        \end{column}
    \end{columns}
    \vspace{0.5cm}
    \begin{alertblock}{Warning}
        Don't force a gap. If the literature already answers your question, adjust your question---don't pretend it doesn't.
    \end{alertblock}
\end{frame}

\begin{frame}{The Research Process}
    \begin{center}
        \begin{tikzpicture}[
            box/.style={rectangle, draw=AccentBlue, fill=LightGray, rounded corners, minimum width=3cm, minimum height=0.8cm, align=center, font=\small},
            arrow/.style={->, thick, AccentBlue}
        ]
            % Discovery phase (left side)
            \node[box] (lit) at (0,0) {Literature};
            \node[box] (questions) at (0,-1.5) {Questions};
            \node[box] (problem) at (0,-3) {Problem};

            % Writing phase (right side)
            \node[box] (question2) at (7,0) {Research Question};
            \node[box] (problem2) at (7,-1.5) {Research Problem};
            \node[box] (solution) at (7,-3) {Solution (Thesis)};

            % Arrows - discovery
            \draw[arrow] (lit) -- (questions);
            \draw[arrow] (questions) -- (problem);

            % Arrows - writing
            \draw[arrow] (question2) -- (problem2);
            \draw[arrow] (problem2) -- (solution);

            % Labels
            \node[font=\bfseries\small, color=AccentBlue] at (0,0.8) {Discovery Phase};
            \node[font=\bfseries\small, color=AccentBlue] at (7,0.8) {Writing Phase};

            % Curved arrow connecting phases
            \draw[arrow, dashed, color=SoftTeal] (problem.east) to[out=0, in=180] (question2.west);
        \end{tikzpicture}
    \end{center}
    \vspace{0.3cm}
    \small
    \textit{Discovery:} Reading generates questions, which reveal problems.\\
    \textit{Writing:} You present question first, then the problem it addresses, then your answer.
\end{frame}

% ============================================
\section{Thesis Structure}
% ============================================

\begin{frame}{Overall Thesis Structure}
    \begin{enumerate}
        \item \textbf{Introduction}
        \begin{itemize}
            \item Research question and problem
            \item Significance and contribution
            \item Thesis outline
        \end{itemize}
        \item \textbf{Literature Review}
        \begin{itemize}
            \item Existing scholarship
            \item Identification of gap
        \end{itemize}
        \item \textbf{Theoretical/Analytical Framework}
        \begin{itemize}
            \item Concepts and approach
        \end{itemize}
        \item \textbf{Methods and Data}
        \begin{itemize}
            \item How you conduct your research
        \end{itemize}
        \item \textbf{Findings/Analysis}
        \begin{itemize}
            \item Your evidence and interpretation
        \end{itemize}
        \item \textbf{Conclusion}
        \begin{itemize}
            \item Summary, implications, limitations
        \end{itemize}
    \end{enumerate}
\end{frame}

\begin{frame}{The Introduction: Your Contract with the Reader}
    \textbf{A strong introduction answers:}
    \vspace{0.3cm}
    \begin{enumerate}
        \item \textbf{WHAT} --- What is your research question?
        \item \textbf{WHY} --- Why does this matter? (the problem/gap)
        \item \textbf{HOW} --- How will you answer it? (brief preview)
        \item \textbf{WHAT NEXT} --- What will the reader find in each chapter?
    \end{enumerate}
    \vspace{0.5cm}
    \begin{block}{Tip}
        Write your introduction last (or revise it heavily at the end). You'll understand your argument much better after writing the whole thesis.
    \end{block}
\end{frame}

% ============================================
\section{Workshop}
% ============================================

\begin{frame}{Self-Reflection Exercise}
    \textbf{Part 1: Brainstorming (10 minutes)}
    \vspace{0.3cm}

    Working from your initial proposal or personal interests:
    \begin{enumerate}
        \item List 2--3 topics you're genuinely curious about
        \item For each topic, write down:
        \begin{itemize}
            \item Why does this interest me personally?
            \item What specific aspect puzzles or intrigues me?
            \item What do I want to understand better?
        \end{itemize}
    \end{enumerate}
    \vspace{0.3cm}
    \begin{block}{Reference}
        See Mullaney \& Rea, pp. 27--28 and 31 for guided self-reflection prompts.
    \end{block}
\end{frame}

\begin{frame}{Drafting Questions}
    \textbf{Part 2: Question Development (10 minutes)}
    \vspace{0.3cm}

    Choose your most promising topic and draft 2--3 possible research questions.
    \vspace{0.3cm}

    \textbf{Check each question against:}
    \begin{itemize}
        \item Is it specific enough to research in 10,000 words?
        \item Is it open-ended (not yes/no)?
        \item Does it avoid causal, leading, or loaded language?
        \item Is it clear to someone outside your specialty?
        \item Does it connect to something you genuinely care about?
    \end{itemize}
\end{frame}

\begin{frame}{Stress-Test Your Questions}
    \textbf{Part 3: Peer Discussion (10 minutes)}
    \vspace{0.3cm}

    With a partner, share your best research question and discuss:
    \begin{enumerate}
        \item Is the question clear and specific?
        \item What evidence would you need to answer it?
        \item What might be the ``problem'' or ``gap'' it addresses?
        \item Is there any problematic language (causal, leading, loaded)?
        \item How could it be improved?
    \end{enumerate}
    \vspace{0.3cm}
    \textbf{Be constructive!} Help each other refine, not just critique.
\end{frame}

% ============================================
\section{Next Steps}
% ============================================

\begin{frame}{Exercise: Draft Your Research Problem}
    \textbf{Due before next class (not graded, but bring to Week 2):}
    \vspace{0.3cm}

    Write approximately 500 words that:
    \begin{enumerate}
        \item \textbf{State the WHAT:}
        \begin{itemize}
            \item Your topic area
            \item Your research question (or 2--3 candidates)
        \end{itemize}
        \item \textbf{State the WHY:}
        \begin{itemize}
            \item Why this matters to you personally
            \item Why it might matter to others (scholars, society)
            \item What gap or problem you think it addresses
        \end{itemize}
    \end{enumerate}
    \vspace{0.3cm}
    \textit{This is exploratory---it's okay if you're uncertain. We'll refine throughout the boot camp.}
\end{frame}

\begin{frame}{Looking Ahead}
    \textbf{Week 2: Data and Sources}
    \begin{itemize}
        \item Primary vs. secondary sources
        \item Qualitative vs. quantitative data
        \item The FAIR principles (Findable, Accessible, Interoperable, Reusable)
        \item Korean-language sources in your research
    \end{itemize}
    \vspace{0.5cm}
    \textbf{Key Dates:}
    \begin{itemize}
        \item Assignment \#1 due: March 13, 2026
        \item Start thinking about your supervisor and schedule a meeting
    \end{itemize}
    \vspace{0.3cm}
    \begin{block}{Course Website}
        \url{https://scdenney.github.io/baks_thesis-seminar/}
    \end{block}
\end{frame}

\end{document}
