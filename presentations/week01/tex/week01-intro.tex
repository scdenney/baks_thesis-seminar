% Week 1: Introduction & Getting Started
% BA2 Digital Korea - Leiden University
% Dr. Steven Denney

\documentclass[aspectratio=169,11pt]{beamer}

% ============================================
% Packages
% ============================================
\usepackage{tikz}
\usepackage{graphicx}
\usepackage{hyperref}
\usepackage{fontspec}
\usepackage{booktabs}

% Load custom theme
\usepackage{beamerthemedigitalkorea}

% ============================================
% Metadata
% ============================================
\title{BA2: Digital Korea}
\subtitle{Week 1: Introduction \& Getting Started}
\author{Steven Denney \& Aron van de Pol}
\institute{Korean Studies \\ Leiden University}
\date{February 2, 2026}

% ============================================
% Document
% ============================================
\begin{document}

% --------------------------------------------
% Title Slide
% --------------------------------------------
\begin{frame}[plain,noframenumbering]
    \titlepage
\end{frame}

% --------------------------------------------
% Outline
% --------------------------------------------
\begin{frame}{Today's Agenda}
    \begin{enumerate}
        \item Welcome \& introductions
        \item Course overview
        \item What is computational text analysis?
        \item Tools and technical setup
        \item In-class assignments
        \item Looking ahead
    \end{enumerate}
\end{frame}

% ============================================
\section{Welcome}
% ============================================

\begin{frame}{About This Course}
    \begin{columns}[T]
        \begin{column}{0.5\textwidth}
            \textbf{The basics}
            \begin{itemize}
                \item 12 sessions
                \item Mondays, 15:15--17:00
                \item Huizinga 0.09 (DH Lab) \& Arsenaal B0.05
            \end{itemize}
            \vspace{0.5cm}
            \textbf{Assessment}
            \begin{itemize}
                \item Participation (15\%)
                \item Research Methods Project (35\%)
                \item Final Paper (50\%)
            \end{itemize}
        \end{column}
        \begin{column}{0.5\textwidth}
            \textbf{What you'll learn}
            \begin{itemize}
                \item Treat text as data
                \item Preprocess, analyze, visualize
                \item Clustering \& classification
                \item Topic modeling
                \item Foundational R programming
            \end{itemize}
        \end{column}
    \end{columns}
\end{frame}

\begin{frame}{About Us}
    \textbf{Dr. Steven Denney}
    \begin{itemize}
        \item Assistant Professor, Korean Studies \& IR
        \item Research: Korean studies, computational social science, DH, comparative politics
        \item \texttt{s.c.denney@hum.leidenuniv.nl}
    \end{itemize}
    \vspace{0.3cm}
    \textbf{Aron van de Pol}
    \begin{itemize}
        \item PhD Candidate, Centre for Digital Humanities (LUCDH)
        \item Research: Korean studies, computer vision, modern Korean print culture
        \item \texttt{a.m.van.de.pol@hum.leidenuniv.nl}
    \end{itemize}
    \vspace{0.3cm}
    \begin{block}{Digital Humanities Lab}
        LUCDH offers support and resources for digital methods in the humanities---a valuable resource for your studies and research.
    \end{block}
\end{frame}

\begin{frame}{About You}
    \begin{center}
        \Large Next week - when I'm there in person!
    \end{center}
\end{frame}

% ============================================
\section{Course Overview}
% ============================================

\begin{frame}{Why Computational Text Analysis?}
    \begin{columns}[T]
        \begin{column}{0.5\textwidth}
            \textbf{The challenge}
            \begin{itemize}
                \item Vast amounts of text data
                \item Historical archives
                \item News, social media, government documents
                \item Too much to read manually
            \end{itemize}
        \end{column}
        \begin{column}{0.5\textwidth}
            \textbf{The opportunity}
            \begin{itemize}
                \item Discover patterns at scale
                \item Systematic, reproducible analysis
                \item New research questions
                \item Complement close reading
            \end{itemize}
        \end{column}
    \end{columns}
    \vspace{0.5cm}
    \begin{alertblock}{Key insight}
        Computational methods don't replace careful reading---they augment it.
    \end{alertblock}
\end{frame}

\begin{frame}{Course Trajectory}
    \begin{center}
        \Large To the course website!
    \end{center}
\end{frame}

\begin{frame}{Learning Objectives}
    By the end of this course, you will be able to:
    \vspace{0.3cm}
    \begin{enumerate}
        \item \textbf{Apply} text preprocessing, descriptive analysis, clustering, classification, and topic modeling
        \item \textbf{Practice} data management and transparency best practices
        \item \textbf{Establish} a foundation in the R programming language
        \item \textbf{Reflect} on the strengths and limitations of computational methods and how they apply to the study of Korea and area studies generally
    \end{enumerate}
\end{frame}

% ============================================
\section{What is Computational Text Analysis?}
% ============================================

\begin{frame}{Text as Data}
    \begin{block}{Core idea}
        Written language can be transformed into structured data that computers can process and analyze.
    \end{block}
    \vspace{0.5cm}
    \textbf{This means:}
    \begin{itemize}
        \item Words become numbers
        \item Meaning becomes measurable
        \item Thousands of texts (``documents") become manageable
        \item Hidden patterns become discoverable
    \end{itemize}
\end{frame}

\begin{frame}{What is a Corpus?}
    \textbf{Corpus} (pl. \textit{corpora}): A structured collection of texts assembled for analysis.
    \vspace{0.5cm}
    \begin{columns}[T]
        \begin{column}{0.5\textwidth}
            \textbf{Examples}
            \begin{itemize}
                \item Presidential speeches
                \item Newspaper articles
                \item Social media posts
                \item Historical documents
                \item Interview transcripts
            \end{itemize}
        \end{column}
        \begin{column}{0.5\textwidth}
            \textbf{Key considerations}
            \begin{itemize}
                \item Selection criteria
                \item Time period
                \item Source(s)
                \item Language(s)
                \item Metadata
            \end{itemize}
        \end{column}
    \end{columns}
\end{frame}

\begin{frame}{Our Corpora}
    \begin{block}{Course materials}
        We will work with curated Korean-language corpora spanning historical texts, periodicals, political speeches, social media, and interview data.
    \end{block}
    \vspace{0.3cm}
    \begin{itemize}
        \item Repository: \url{https://github.com/scdenney/nlp_corpora}
        \item Focus on Korea-relevant content
        \item Truncated versions for classroom use
        \item Accommodations for non-Korean readers
        \item We will grow this repository
    \end{itemize}
\end{frame}

% ============================================
\section{Tools \& Technical Setup}
% ============================================

\begin{frame}{Our Toolkit}
    \begin{columns}[T]
        \begin{column}{0.5\textwidth}
            \textbf{Primary analysis}
            \begin{itemize}
                \item \textbf{Orange Data Mining}
                \item Visual, drag-and-drop interface
                \item No programming required
                \item Powerful text analysis widgets
            \end{itemize}
        \end{column}
        \begin{column}{0.5\textwidth}
            \textbf{Programming foundation}
            \begin{itemize}
                \item \textbf{R + RStudio}
                \item Industry-standard for data science
                \item \textbf{Swirl} for interactive tutorials
                \item \textbf{DataCamp} for guided courses
            \end{itemize}
        \end{column}
    \end{columns}
\end{frame}

\begin{frame}{Why Orange Data Mining?}
    \begin{itemize}
        \item Widget-based: Build workflows visually
        \item Accessible: Focus on concepts, not coding
        \item Powerful: Real analysis capabilities
    \end{itemize}
\end{frame}

\begin{frame}{Why R?}
    \begin{columns}[T]
        \begin{column}{0.5\textwidth}
            \textbf{Practical reasons}
            \begin{itemize}
                \item Free and open source
                \item Huge ecosystem of packages
                \item Strong text analysis tools
                \item Reproducible research
            \end{itemize}
        \end{column}
        \begin{column}{0.5\textwidth}
            \textbf{Career reasons}
            \begin{itemize}
                \item Widely used in academia
                \item Growing in industry
                \item Transferable skill
                \item Gateway to Python, etc.
            \end{itemize}
        \end{column}
    \end{columns}
    \vspace{0.5cm}
    \begin{exampleblock}{Note}
        No prior programming experience required. We learn together.
    \end{exampleblock}
\end{frame}

\begin{frame}{GitHub for Version Control}
    \textbf{Why GitHub?}
    \begin{itemize}
        \item Track changes to your work
        \item Collaborate and share
        \item Industry-standard workflow
        \item Portfolio for future work
    \end{itemize}
    \vspace{0.5cm}
    \textbf{Course website:} \url{https://scdenney.github.io/ba2_digital-korea}
\end{frame}

% ============================================
\section{In-Class Assignments}
% ============================================

\begin{frame}{Today's Tasks}
    \begin{alertblock}{Do today, from "Getting Started"}
        These steps ensure you have the technical foundation for the semester.
    \end{alertblock}
    \vspace{0.3cm}
    \begin{enumerate}
        \item GitHub setup
        \item Create class repo and share with 'scdenney' (that's me)
        \item Confirm DataCamp enrollment (check email)
        \item Verify installations: RStudio, Swirl, Orange Data Mining (next slide)
    \end{enumerate}
\end{frame}

\begin{frame}{Software Verification}
    \textbf{Check that you have installed:}
    \vspace{0.3cm}
    \begin{itemize}
        \item[$\square$] \textbf{R} --- \url{r-project.org}
        \item[$\square$] \textbf{RStudio} --- \url{posit.co/download/rstudio-desktop}
        \item[$\square$] \textbf{Swirl} --- Run in R: \texttt{install.packages("swirl")}
        \item[$\square$] \textbf{Orange Data Mining} --- \url{orangedatamining.com/download}
    \end{itemize}
    \vspace{0.5cm}
    \begin{block}{Trouble?}
        Not to worry. We'll troubleshoot together.
    \end{block}
\end{frame}

% ============================================
\section{Looking Ahead}
% ============================================

\begin{frame}{For Next Week}
    \textbf{R Programming (due by start of Week 2):}
    \begin{itemize}
        \item Complete Swirl R Programming lessons:
        \begin{itemize}
            \item 1: Basic Building Blocks
            \item 2: Workspace and Files
            \item 4: Vectors
            \item 6: Subsetting Vectors
            \item 7: Matrices and Data Frames
            \item 12: Looking at Data
        \end{itemize}
    \end{itemize}
    \vspace{0.3cm}
    \textbf{Week 2 topic:} Foundations of Computational Text Analysis
\end{frame}

\begin{frame}{For Next Week}
    \textbf{For Week 2:}
    \vspace{0.3cm}
    \begin{itemize}
        \item Grimmer, Roberts, and Stewart --- Chapter 2: ``Social Science Research and Text Analysis'' (will be provided via email)
        \item Markdown explainer (on course website, under ``Getting Started'')
    \end{itemize}
    \vspace{0.5cm}
    \textbf{Orange Data Mining Tutorials:}
    \begin{itemize}
        \item Getting Started 01--04
        \item \url{https://www.youtube.com/playlist?list=PLmNPvQr9Tf-ZSDLwOzxpvY-HrE0yv-8Fy}
    \end{itemize}
    \vspace{0.5cm}
    \textbf{Bookmark this:}
    \begin{itemize}
        \item Orange Widget Catalog: \url{https://orangedatamining.com/widget-catalog/}
        \item Your go-to reference when learning new widgets or troubleshooting
    \end{itemize}
\end{frame}

\end{document}
